\documentclass[a4paper,11pt]{article}

%%%%%%%% CREATE DOCUMENT STRUCTURE %%%%%%%%
%% Language and font encodings
\usepackage[italian]{babel}
\usepackage[utf8x]{inputenc}
\usepackage[T1]{fontenc}

%% Sets page size and margins
\usepackage[a4paper,top=3cm,bottom=2cm,left=2cm,right=2cm,marginparwidth=1.75cm]{geometry}

%% Useful packages
\usepackage{amsmath}
\usepackage{amsfonts} % for /mathbb and similar math symbols
\usepackage{graphicx}
\usepackage[colorlinks=true, allcolors=blue]{hyperref}
\usepackage{caption}
\usepackage{subcaption}
\usepackage{float}
\usepackage{titling}
\usepackage{blindtext}
\usepackage[square,sort,comma,numbers]{natbib}
\usepackage{xcolor}

\newcommand{\HRule}{\rule{\linewidth}{0.5mm}} 	% horizontal line and its thickness

% Functions defined by two cases
\newcommand{\twopartdef}[4]
    {
        \left\{
            \begin{array}{ll}
                #1 & \mbox{if } #2 \\
                #3 & \mbox{if } #4
            \end{array}
        \right.
    }
% Functions defined by three cases
\newcommand{\threepartdef}[6]
{
    \left\{
	\begin{array}{lll}
	    	#1 & \mbox{if } #2 \\
		#3 & \mbox{if } #4 \\
		#5 & \mbox{if } #6
	\end{array}
    \right.
}


%%%%%%%% DOCUMENT %%%%%%%%
\begin{document}

%%%% Title Page
\begin{titlepage}
\center
% Some logo, optional
\includegraphics[width=0.6\textwidth]{img/unipi-logo.png}\\[1cm]

% University
\textsc{\LARGE Università di Pisa, Dipartimento di Informatica}\\[1cm]

% Document info
\textsc{\Large Laboratorio di Basi di dati - A.A. 2021-2022}\\[0.2cm]
\textsc{\large Docente: Giovanna Rosone}\\[1cm]

% Assignment title (enclosed between horizontal lines
\HRule \\[0.8cm]
{ \huge \bfseries Specifica delle operazioni}\\[0.7cm]
\HRule \\[2cm]

% Author and date
\Huge \emph{Gruppo 2}\\[0.5cm]
\large Francesco Nicolò\\Roberto Puggioni\\Jacopo Raffi\\Nicola Vetrini\\[1.5cm]
{\large \today}\\[5cm]

\vfill
\end{titlepage}

\newpage
\tableofcontents
\newpage

%%%%%%%%%%%%%%%%%%%%%%%%%%%
% Operazioni sulle Opere  %
%%%%%%%%%%%%%%%%%%%%%%%%%%%
\section{Operazioni sulle Opere}

%%%%%%%%%%%%%%%%%%%%%%%%%%%
%     Visualizza Opera    %
%%%%%%%%%%%%%%%%%%%%%%%%%%%
\subsection{Visualizzazione}
\begin{itemize}
	\item \textbf{Operazione}: VisualizzaOpera
	\item \textbf{Obiettivo}: Visualizzare le informazioni relative all'Opera
	\item \textbf{Parametri}:
		\begin{enumerate}
			\item \textbf{idSessione}: in NUMBER DEFAULT 0
			\item \textbf{operaID}: in NUMBER DEFAULT 0
			\item \textbf{lingue}: in VARCHAR2 DEFAULT 'sconosciuto'
			\item \textbf{livelli}: in VARCHAR2 DEFAULT 'sconosciuto'
		\end{enumerate}
	\item \textbf{Risultato}: Visualizza l'Opera richiesta
	\item \textbf{Errori}: l'Opera con operaID specificato non è presente nel database
	\item \textbf{Usa}: Opere
	\item \textbf{Modifica}: nessuna
	\item \textbf{Precondizioni}:
		\begin{itemize}
			\item operaID è la chiave primaria di una ennupla in Opere
		\end{itemize}
	\item \textbf{Postcondizioni}: nessuna
\end{itemize}

%%%%%%%%%%%%%%%%%%%%%%%%%%%
%     Inserisci Opera     %
%%%%%%%%%%%%%%%%%%%%%%%%%%%
\subsection{Inserimento}
\begin{itemize}
	\item \textbf{Operazione}: InserisciOpera
	\item \textbf{Obiettivo}: Inserisce una nuova Opera nel database
	\item \textbf{Parametri}:
	\begin{enumerate}
		\item \textbf{idSessione}: in (da definire)
		\item \textbf{MuseoID}: in number(5)
		\item \textbf{Titolo}: in varchar2(15)
		\item \textbf{Anno}: in date
	\end{enumerate}
	\item \textbf{Risultato}: Inserisce l'Opera richiesta nella base di dati, oppure restituisce un errore
	\item \textbf{Errori}: 
	\begin{itemize}
		\item L'Opera con i dati inseriti è già presente nel database
		\item Non esiste alcun Museo con IdMuseo = MuseoID
	\end{itemize}
	\item \textbf{Usa}: Opere, Musei, AppartieneA
	\item \textbf{Modifica}: Opere
	\item \textbf{Precondizioni}:
	\begin{itemize}
		\item $MuseoID \ne null \land Titolo \ne null \land Anno \ne null$
		\item $\exists x \in Musei : x.IdMuseo = MuseoID$
		\item $\not\exists x \in Opere : x.Museo = MuseoID \land x.Titolo = Titolo \land x.Anno = Anno$
		\item $|Opere| = n$
	\end{itemize}
	\item \textbf{Postcondizioni}: inserita l'ennupla (IdOpera,Titolo,Anno,MuseoID) in Opere
\end{itemize}

%%%%%%%%%%%%%%%%%%%%%%%%%%%
%     Modifica Opera      %
%%%%%%%%%%%%%%%%%%%%%%%%%%%
\subsection{Modifica}
\begin{itemize}
	\item \textbf{Operazione}: ModificaOpera
	\item \textbf{Obiettivo}: Modifica un'Opera presente nel database
	\item \textbf{Parametri}:
	\begin{enumerate}
		\item \textbf{idSessione}: in (da definire)
		\item \textbf{OperaID}: in number(5)
		\item \textbf{newtitle}: in varchar2(15) default null
		\item \textbf{newyear}: in date default null
		\item \textbf{newmuseum}: in number(5) default null
	\end{enumerate}
	\item \textbf{Risultato}: Modifica l'Opera specificata se esiste nella base di dati, errore altrimenti (la base di dati rimane inalterata)
	\item \textbf{Errori}: 
	\begin{itemize}
		\item L'Opera con OperaID non è presente nel database
	\end{itemize}
	\item \textbf{Usa}: Opere, Musei
	\item \textbf{Modifica}: Opere
	\item \textbf{Precondizioni}:
	\begin{itemize}
		\item $\exists x \in Opere : x.IdOpera = OperaID$
		\item $(newmuseum \ne null) \lor (\exists x \in Musei : x.IdMuseo = newmuseum)$
		\item $|Opere| = n$
	\end{itemize}
	\item \textbf{Postcondizioni}: se $x \in Opere \land x.IdOpera = OperaID$
	\begin{itemize}
		\item Se $newtitle \ne null \Rightarrow x.Titolo := newtitle$
		\item Se $newyear \ne null \Rightarrow x.Anno := newyear$
		\item Se $newmuseum \ne null \Rightarrow x.Museo := newmuseum$
		\item $|Opere| = n$
	\end{itemize}
	\item Note: Se uno dei parametri opzionali non è specificato allora tale attributo 
	dell'Opera non viene modificato
\end{itemize}

%%%%%%%%%%%%%%%%%%%%%%%%%%%
%     Rimuovi Opera       %
%%%%%%%%%%%%%%%%%%%%%%%%%%%
\subsection{Rimozione}
\begin{itemize}
	\item \textbf{Operazione}: RimuoviOpera
	\item \textbf{Obiettivo}: Rimuove un'Opera dal database
	\item \textbf{Parametri}:
	\begin{enumerate}
		\item \textbf{idSessione}: in (da definire)
		\item \textbf{OperaID}: in number(5)
	\end{enumerate}
	\item \textbf{Risultato}: Rimuove l'Opera specificata se esiste nella base di dati, errore altrimenti (la base di dati rimane inalterata)
	\item \textbf{Errori}: 
	\begin{itemize}
		\item L'Opera con OperaID non è presente nel database
	\end{itemize}
	\item \textbf{Usa}: Opere
	\item \textbf{Modifica}: Opere
	\item \textbf{Precondizioni}:
	\begin{itemize}
		\item $\exists x \in Opere : x.IdOpera = OperaID$
		\item $|Opere| = n$
	\end{itemize}
	\item \textbf{Postcondizioni}:
	\begin{itemize}
		\item $\not\exists x \in Opere : x.IdOpera = OperaID$
		\item $|Opere| = n - 1$
	\end{itemize}
	\item Note: Se uno dei parametri opzionali non è specificato allora tale attributo 
	dell'Opera non viene modificato
\end{itemize}

%%%%%%%%%%%%%%%%%%%%%%%%%%%
%  Aggiunta Autore Opera  %
%%%%%%%%%%%%%%%%%%%%%%%%%%%
% Si intende l'aggiunta di un autore già esistente nella base di dati
\subsection{Aggiunta Autore} \large \colorbox{red}{TODO: decidere se idempotente}
\begin{itemize}
	\item \textbf{Operazione}: AggiungiAutoreOpera
	\item \textbf{Obiettivo}: Aggiunge un Autore ad un'Opera, se entrambe le entità sono presenti nella base di dati
	\item \textbf{Parametri}:
	\begin{enumerate}
		\item \textbf{idSessione}: in (da definire)
		\item \textbf{OperaID}: in number(5)
		\item \textbf{AuthorID}: in number(5)
	\end{enumerate}
	\item \textbf{Risultato}: Aggiunge l'autore con l'ID indicato all'Opera individuata dall'ID dato.\\
	Se l'autore o l'opera non sono presenti allora viene restituito errore.	Viene restituito errore anche se è già stato aggiunto l'autore specificato a questa Opera
	\item \textbf{Errori}: 
	\begin{itemize}
		\item L'Opera con OperaID non è presente nel database
		\item L'autore con AuthorID non è presente nel database
		\item L'autore individuato da AuthorID è già stato aggiunto come autore di quest'Opera
	\end{itemize}
	\item \textbf{Usa}: Opere, Autori, AutoriOpere
	\item \textbf{Modifica}: AutoriOpere
	\item \textbf{Precondizioni}:
	\begin{itemize}
		\item $\exists x \in Opere : x.IdOpera = OperaID$
		\item $\exists y \in Autori : y.IdAutore = AuthorID$
		\item $\not\exists z \in AutoriOpere : z.IdAutore = AuthorID \land z.IdOpera = OperaID$
		\item $|AutoriOpere| = n$
	\end{itemize}
	\item \textbf{Postcondizioni}:
	\begin{itemize}
		\item $\exists z \in AutoriOpere : z.IdAutore = AuthorID \land z.IdOpera = OperaID$
		\item $|AutoriOpere| = n + 1$
	\end{itemize}
	\item \textbf{Note}: L'operazione non è idempotente \textbf{\color{red} Da decidere se renderla tale}
\end{itemize}

\subsection{Rimozione Autore}
\begin{itemize}
	\item \textbf{Operazione}: AggiungiAutoreOpera
	\item \textbf{Obiettivo}: Aggiunge un Autore ad un'Opera, se entrambe le entità sono presenti nella base di dati
	\item \textbf{Parametri}:
	\begin{enumerate}
		\item \textbf{idSessione}: in (da definire)
		\item \textbf{OperaID}: in number(5)
		\item \textbf{AuthorID}: in number(5)
	\end{enumerate}
	\item \textbf{Risultato}: Aggiunge l'autore con l'ID indicato all'Opera individuata dall'ID dato.\\
	Se l'autore o l'opera non sono presenti allora viene restituito errore.	Viene restituito errore anche se è già stato aggiunto l'autore specificato a questa Opera
	\item \textbf{Errori}: 
	\begin{itemize}
		\item L'Opera con OperaID non è presente nel database
		\item L'autore con AuthorID non è presente nel database
		\item L'autore individuato da AuthorID è già stato aggiunto come autore di quest'Opera
	\end{itemize}
	\item \textbf{Usa}: Opere, Autori, AutoriOpere
	\item \textbf{Modifica}: AutoriOpere
	\item \textbf{Precondizioni}:
	\begin{itemize}
		\item $\exists x \in Opere : x.IdOpera = OperaID$
		\item $\exists y \in Autori : y.IdAutore = AuthorID$
		\item $\not\exists z \in AutoriOpere : z.IdAutore = AuthorID \land z.IdOpera = OperaID$
		\item $|AutoriOpere| = n$
	\end{itemize}
	\item \textbf{Postcondizioni}:
	\begin{itemize}
		\item $\exists z \in AutoriOpere : z.IdAutore = AuthorID \land z.IdOpera = OperaID$
		\item $|AutoriOpere| = n + 1$
	\end{itemize}
	\item \textbf{Note}: L'operazione non è idempotente \textbf{\color{red} Da decidere se renderla tale}
\end{itemize}

\subsection{Statistiche e monitoraggio}

\subsubsection{Storico prestiti dell’Opera}
\subsubsection{Autori dell’Opera}
\subsubsection{Tipo Sala in cui si trova l’Opera}
\subsubsection{Descrizioni dell’Opera}
\subsubsection{Lista Opere ordinate per numero di Autori in ordine decrescente}
\subsubsection{Opere non spostata da più tempo (le tre più vecchie)}
\subsubsection{Opere esposte per più tempo (le tre più vecchie)}
\subsubsection{Età media delle opere}
\subsubsection{Ordinamento per anno di realizzazione (le tre più vecchie)}

\newpage

%%%%%%%%%%%%%%%%%%%%%%%%%%%%%%%
%   Operazioni sugli Autori   %
%%%%%%%%%%%%%%%%%%%%%%%%%%%%%%%
\section{Operazioni sugli Autori}

%%%%%%%%%%%%%%%%%%%%%%%%%%%%%%
%     Visualizza Autore      %
%%%%%%%%%%%%%%%%%%%%%%%%%%%%%%
\subsection{Visualizzazione}
\begin{itemize}
	\item \textbf{Operazione}: VisualizzaAutore
	\item \textbf{Obiettivo}: Visualizza le informazioni relative ad un Autore presente nel database
	\item \textbf{Parametri}:
	\begin{enumerate}
		\item \textbf{idSessione}: in (da definire)
		\item \textbf{AuthorID}: in number(5)
	\end{enumerate}
	\item \textbf{Risultato}: Visualizza i dati relativi all'autore identificato da AuthorID, se presente nel database, restituisce un errore altrimenti
	\item \textbf{Errori}: 
	\begin{itemize}
		\item L'Autore identificato da AuthorID non è presente nel database
	\end{itemize}
	\item \textbf{Usa}: Autori
	\item \textbf{Modifica}: nessuno
	\item \textbf{Precondizioni}:
	\begin{itemize}
		\item $\exists x \in Autori : x.IdAutore = AuthorID$
	\end{itemize}
	\item \textbf{Postcondizioni}: nessuna
\end{itemize}

%%%%%%%%%%%%%%%%%%%%%%%%%%%%%%
%      Inserisci Autore      %
%%%%%%%%%%%%%%%%%%%%%%%%%%%%%%
\subsection{Inserimento}
\begin{itemize}
	\item \textbf{Operazione}: InserisciAutore
	\item \textbf{Obiettivo}: Inserisce un nuovo autore nel database
	\item \textbf{Parametri}:
	\begin{enumerate}
		\item \textbf{idSessione}: in (da definire)
		\item \textbf{nome}: in varchar2(25)
		\item \textbf{cognome}: in varchar2(25)
		\item \textbf{dataNascita}: in date
		\item \textbf{dataMorte}: in date
		\item \textbf{nazionalita}: in varchar2(25)
	\end{enumerate}
	\item \textbf{Risultato}: Inserisce un nuovo autore nella tabella Autori con i dati specificati dai parametri
	\item \textbf{Errori}: 
	\begin{itemize}
		\item È già presente nel database un autore con gli stessi dati
	\end{itemize}
	\item \textbf{Usa}: Autori
	\item \textbf{Modifica}: Autori
	\item \textbf{Precondizioni}:
	\begin{itemize}
		\item $nome \ne null \land cognome \ne null \land nazionalita \ne null$
		\item $\not\exists x \in Autori : x.Nome = nome \land x.Cognome = congnome 
		\land x.DataNascita = dataNascita \land x.DataMorte = dataMorte 
		\land x.Nazionalita = nazionalita$
		\item $|Autori| = n$
	\end{itemize}
	\item \textbf{Postcondizioni}:
	\begin{itemize}
		\item $(IdAutore,nome,cognome,dataNascita,dataMorte,nazionalita) \in Autori$
		\item $|Autori| = n + 1$
	\end{itemize}
\end{itemize}

%%%%%%%%%%%%%%%%%%%%%%%%%%%%%%
%      Modifica Autore      %
%%%%%%%%%%%%%%%%%%%%%%%%%%%%%%
\subsection{Modifica}
\begin{itemize}
	\item \textbf{Operazione}: ModificaAutore
	\item \textbf{Obiettivo}: Modifica i dati di un Autore presente nel database
	\item \textbf{Parametri}:
	\begin{enumerate}
		\item \textbf{idSessione}: in (da definire)
		\item \textbf{AuthorID}: in number(5)
		\item \textbf{newname}: in varchar2(25) default null
		\item \textbf{newsurname}: in varchar2(25) default null
		\item \textbf{newbirth}: in date default null
		\item \textbf{newdeath}: in date default null
		\item \textbf{newnation}: in varchar2(25) default null
	\end{enumerate}
	\item \textbf{Risultato}: Modifica i dati di un Autore nella base di dati, oppure restituisce un errore (e non modifica la base di dati)
	\item \textbf{Errori}: 
	\begin{itemize}
		\item L'autore da modificare non è presente nella base di dati
	\end{itemize}
	\item \textbf{Usa}: Autori
	\item \textbf{Modifica}: Autori
	\item \textbf{Precondizioni}:
	\begin{itemize}
		\item $\exists x \in Autori : x.IdAutore = AuthorID$
		\item $|Autori| = n$
	\end{itemize}
	\item \textbf{Postcondizioni}: $x \in Autori : x.IdAutore = AuthorID$
	\begin{itemize}
		\item Se $newname \ne null \Rightarrow x.Nome := newname$
		\item Se $newsurname \ne null \Rightarrow x.Cognome := newsurname$
		\item Se $newbirth \ne null \Rightarrow x.DataNascita := newbirth$
		\item Se $newdeath \ne null \Rightarrow x.DataMorte := newdeath$
		\item Se $newnation \ne null \Rightarrow x.Nazionalita := newnation$
		\item $|Autori| = n$
	\end{itemize}
\end{itemize}


%%%%%%%%%%%%%%%%%%%%%%%%%%%%%%
%       Rimuovi Autore       %
%%%%%%%%%%%%%%%%%%%%%%%%%%%%%%
\subsection{Rimozione}
\begin{itemize}
	\item \textbf{Operazione}: RimuoviAutore
	\item \textbf{Obiettivo}: Rimuove un Autore dal database, se presente
	\item \textbf{Parametri}:
	\begin{enumerate}
		\item \textbf{idSessione}: in (da definire)
		\item \textbf{AuthorID}: in number(5)
	\end{enumerate}
	\item \textbf{Risultato}: Rimuove un Autore dalla base di dati, oppure restituisce un errore (e non modifica la base di dati)
	\item \textbf{Errori}: 
	\begin{itemize}
		\item L'autore da rimuovere non è presente nella base di dati
	\end{itemize}
	\item \textbf{Usa}: Autori
	\item \textbf{Modifica}: Autori
	\item \textbf{Precondizioni}:
	\begin{itemize}
		\item $\exists x \in Autori : x.IdAutore = AuthorID$
		\item $|Autori| = n$
	\end{itemize}
	\item \textbf{Postcondizioni}:
	\begin{itemize}
		\item $\not\exists x \in Autori : x.IdAutore = AuthorID$
		\item $|Autori| = n - 1$
	\end{itemize}
\end{itemize}

%%%%%%%%%%%%%%%%%%%%%%%%%%%%%%
%       Ricerca Autore       %
%%%%%%%%%%%%%%%%%%%%%%%%%%%%%%
\subsection{Ricerca}
\begin{itemize}
	\item \textbf{Operazione}: RicercaAutore
	\item \textbf{Obiettivo}: Cerca un Autore nel database (settare un parametro a null ha la semantica di non filtrare gli autori risultanti sulla base di tale parametro)
	\item \textbf{Parametri}:
	\begin{enumerate}
		\item \textbf{idSessione}: in (da definire)
		\item \textbf{srcNome}: in varchar2(25) default null
		\item \textbf{srcCognome}: in varchar2(25) default null
		\item \textbf{srcDataNascita}: in date default null
		\item \textbf{srcDataMorte}: in date default null
		\item \textbf{srcNazionalita}: in varchar2(25) default null
	\end{enumerate}
	\item \textbf{Risultato}: Ritorna la lista di autori nella base di dati che rispettano i criteri di ricerca (anche nessun autore, quindi una lista vuota)
	\item \textbf{Errori}: 
	\begin{itemize}
		\item nessuno
	\end{itemize}
	\item \textbf{Usa}: Autori
	\item \textbf{Modifica}: nessuna
	\item \textbf{Precondizioni}: nessuna
	\item \textbf{Postcondizioni}:
 	\begin{align*}
		(\forall x \in ListaAutori.
		& (srcNome \ne null \Rightarrow srcNome \subseteq x.Nome) \\
		& \land (srcNome \ne null \Rightarrow srcCogome \subseteq x.Cognome) \\
		& \land (srcDataNascita \ne null \Rightarrow srcDataNascita \le x.DataNascita) \\
		& \land (srcDataMorte \ne null \Rightarrow srcDataMorte \ge x.DataMorte) \\
		& \land (srcNazionalita \ne null \Rightarrow srcNazionalita = x.Nazionalita))
	\end{align*}
\end{itemize}

\subsection{Statistiche e monitoraggio}
\subsubsection{Opere realizzate dall’Autore}
\subsubsection{Musei con Opere dell’Autore esposte}
\subsubsection{Collaborazioni effettuate (TODO)}
\subsubsection{Opere dell’Autore presenti in un Museo scelto}
\subsubsection{Autori in vita le cui Opere sono esposte in un Museo scelto}

%%%%%%%%%%%%%%%%%%%%%%%%%%%%%%%%
% Operazioni sulle Descrizioni %
%%%%%%%%%%%%%%%%%%%%%%%%%%%%%%%%
\section{Operazioni sulle Descrizioni}

%%%%%%%%%%%%%%%%%%%%%%%%%%%%%%
%   Visualizza Descrizione   %
%%%%%%%%%%%%%%%%%%%%%%%%%%%%%%
\subsection{Visualizzazione}
\begin{itemize}
	\item \textbf{Operazione}: visualizzaDescrizione
	\item \textbf{Obiettivo}: Visualizza la Descrizione di un'Opera, dato il suo IdDescr
	\item \textbf{Parametri}:
	\begin{enumerate}
		\item \textbf{idSessione}: in (da definire)
		\item \textbf{DescrID}: in number(5)
	\end{enumerate}
	\item \textbf{Risultato}: Visualizza i dettagli della Descrizione richiesta, altrimenti restituisce errore
	\item \textbf{Errori}: 
	\begin{itemize}
		\item La descrizione richiesta non è presente nella base di dati
	\end{itemize}
	\item \textbf{Usa}: Descrizioni
	\item \textbf{Modifica}: Descrizioni
	\item \textbf{Precondizioni}:
	\begin{itemize}
		\item $\exists x \in Descrizioni : x.IdDescr = DescrID$
	\end{itemize}
	\item \textbf{Postcondizioni}: nessuna
\end{itemize}

%%%%%%%%%%%%%%%%%%%%%%%%%%%%%%
%   Inserisci Descrizione   %
%%%%%%%%%%%%%%%%%%%%%%%%%%%%%%
\subsection{Inserimento}
\begin{itemize}
	\item \textbf{Operazione}: inserisciDescrizione
	\item \textbf{Obiettivo}: Inserisce una nuova descrizione, associandola ad un'Opera
	\item \textbf{Parametri}:
	\begin{enumerate}
		\item \textbf{idSessione}: in (da definire)
		\item \textbf{lingua}: in varchar2(25)
		\item \textbf{livello}: in varchar2(25)
		\item \textbf{testodescr}: in CLOB
		\item \textbf{OperaID}: in number(5)
	\end{enumerate}
	\item \textbf{Risultato}: Inserisce la descrizione dell'Opera, restituisce errore se l'Opera non è presente nel database
	\item \textbf{Errori}: 
	\begin{itemize}
		\item L'Opera a cui associare la descrizione non è presente nella base di dati
	\end{itemize}
	\item \textbf{Usa}: Descrizioni, Opere
	\item \textbf{Modifica}: Descrizioni
	\item \textbf{Precondizioni}:
	\begin{itemize}
		\item $\exists x \in Opere : x.IdOpera = OperaID$
	\end{itemize}
	\item \textbf{Postcondizioni}:
	\begin{itemize}
		\item $\exists x \in Descrizioni : x.Opera = OperaID \land x.Lingua = lingua \land x.Livello = livello \land x.Testo = testodescr$
	\end{itemize}
\end{itemize}

%%%%%%%%%%%%%%%%%%%%%%%%%%%%%%
%    Modifica Descrizione    %
%%%%%%%%%%%%%%%%%%%%%%%%%%%%%%
\subsection{Modifica}
\begin{itemize}
	\item \textbf{Operazione}: ModificaDescrizione
	\item \textbf{Obiettivo}: Modifica la descrizione di un'opera (è possibile modificare anche l'opera a cui una descrizione è associata)
	\item \textbf{Parametri}:
	\begin{enumerate}
		\item \textbf{idSessione}: in (da definire)
		\item \textbf{descrID}: in number(5)
		\item \textbf{newlang}: in varchar2(25) default null
		\item \textbf{newlevel}: in varchar2(25) default null
		\item \textbf{newtext}: in CLOB default null
		\item \textbf{newOpera}: in number(5) default null
	\end{enumerate}
	\item \textbf{Risultato}: Modifica gli attributi della descrizione individuata da descrID, oppure ritorna un errore
	\item \textbf{Errori}: 
	\begin{itemize}
		\item La descrizione da modificare non è presente nella base di dati
		\item La nuova Opera da associare alla descrizione non è presente nella base di dati
	\end{itemize}
	\item \textbf{Usa}: Opere, Descrizioni
	\item \textbf{Modifica}: Descrizioni
	\item \textbf{Precondizioni}:
	\begin{itemize}
		\item $\exists x \in Descrizioni : x.IdDescr = descrID$
		\item $\exists x \in Opere : x.IdOpera = newOpera$
	\end{itemize}
	\item \textbf{Postcondizioni}:
		\begin{align*} (\exists x \in Descrizioni.
		& (newlang \ne null \Rightarrow x.Lingua = newlang) \\
		& \land (newlevel \ne null \Rightarrow x.Livello = newlevel) \\
		& \land (newtext \ne null \Rightarrow x.Testo = newtext) \\
		& \land (newOpera \ne null \Rightarrow x.Opera = newOpera))
		\end{align*}
\end{itemize}

%%%%%%%%%%%%%%%%%%%%%%%%%%%%%%
%   Rimozione Descrizione    %
%%%%%%%%%%%%%%%%%%%%%%%%%%%%%%
\subsection{Rimozione}
\begin{itemize}
	\item \textbf{Operazione}: rimuoviDesrizione
	\item \textbf{Obiettivo}: Rimuove una descrizione, se presente
	\item \textbf{Parametri}:
	\begin{enumerate}
		\item \textbf{idSessione}: in (da definire)
		\item \textbf{descrID}: in number(5)
	\end{enumerate}
	\item \textbf{Risultato}: Rimuove la descrizione dell'Opera, restituendo errore se la descrizione non è presente nel database
	\item \textbf{Errori}: 
	\begin{itemize}
		\item La Descrizione identificata da descrID non è presente nella base di dati
	\end{itemize}
	\item \textbf{Usa}: Descrizioni
	\item \textbf{Modifica}: Descrizioni
	\item \textbf{Precondizioni}:
	\begin{itemize}
		\item $\exists x \in Descrizioni : x.IdDescr = descrID$
		\item $|Descrizioni| = n$
	\end{itemize}
	\item \textbf{Postcondizioni}:
	\begin{itemize}
		\item $\not\exists x \in Descrizioni : x.IdDescr = descrID$
		\item $|Descrizioni| = n - 1$
	\end{itemize}
\end{itemize}

\subsection{Statistiche e monitoraggio}
\subsubsection{Livello descrizione più presente}
\subsubsection{Lingua più presente}

\end{document}
